\section{Numerical Sequences and Series}

\begin{questions}
  \question Prove that convergence of $\{s_n\}$ implies convergence of $\{\abs{s_n}\}$. Is the converse true?

  \question Calculate $\lim_{n\to\infty}(\sqrt{n^2+n}-n)$.

  \question If $s_1=2$, and
  \[ s_{n+1} = \sqrt{2+\sqrt{s_n}} \qquad (n=1,2,3,\ldots), \]
  prove that $\{s_n\}$ converges, and that $s_n<2$ for $n=1,2,3\ldots.$

  \question Find the upper and lower limits of the sequence $\{s_n\}$ defined by
  \[ s_1 = 0; \qquad s_{2m} = \frac{s_{2m-1}}{2}; \qquad s_{2m+1} = \frac{1}{2} + s_{2m}. \]

  \question For any two real sequences $\{a_n\}$, $\{b_n\}$, prove that
  \[ \limsup_{n\to\infty} (a_n+b_n) \leq \limsup_{n\to\infty} a_n + \limsup_{n\to\infty} b_n, \]
  provided the sum on the right is not of the form $\infty-\infty$.

  \question Investigate the behaviour (convergence or divergence) of $\sum a_n$ if
  \begin{parts}
    \part $a_n=\sqrt{n+1}-\sqrt{n}$;

    \part $a_n=\dfrac{\sqrt{n+1}-\sqrt{n}}{n}$;

    \part $a_n=(\sqrt[n]{n}-1)^n$;

    \part $a_n=\dfrac{1}{1+z^n}$, \qquad for complex values of $z$.
  \end{parts}

  \question Prove that the convergence of $\sum a_n$ implies the convergence of
  \[ \sum \frac{\sqrt{a_n}}{n} \]
  if $a_n\geq0$.

  \question If $\sum a_n$ converges, and if $\{b_n\}$ is monotonic and bounded, prove that $\sum a_nb_n$ converges.

  \question Find the radius of convergence of each of the following power series:
  \begin{parts}
    \part $\sum n^3z^n$,

    \part $\sum\dfrac{2^n}{n!}z^n$,

    \part $\sum\dfrac{2^n}{n^2}z^n$,

    \part $\sum\dfrac{n^3}{3^n}z^n$.
  \end{parts}

  \question Suppose that the coefficients of the power series $\sum a_nz^n$ are integers, infinitely many of which are distinct from zero. Prove that the radius of convergence is at most 1.

  \question Suppose $a_n>0$, $s_n=a_1+\cdots+a_n$, and $\sum a_n$ diverges.
  \begin{parts}
    \part Prove that $\sum\dfrac{a_n}{1+a_n}$ diverges.

    \part Prove that
    \[ \frac{a_{N+1}}{s_{N+1}} + \cdots + \frac{a_{N+k}}{s_{N+k}} \geq 1 - \frac{s_N}{s_{N+k}} \]
    and deduce that $\sum\dfrac{a_n}{s_n}$ diverges.

    \part Prove that
    \[ \frac{a_n}{s_n^2} \leq \frac{1}{s_{n-1}} - \frac{1}{s_n} \]
    and deduce that $\sum\dfrac{a_n}{s_n^2}$ converges.

    \part What can be said about
    \[ \sum \frac{a_n}{1+na_n} \quad \text{and} \quad \sum\frac{a_n}{1+n^2a_n}? \]
  \end{parts}

  \question Suppose $a_n>0$ and $\sum a_n$ converges. Put
  \[ r_n = \sum_{m=n}^\infty a_m. \]
  \begin{parts}
    \part Prove that
    \[ \frac{a_m}{r_m} + \cdots + \frac{a_n}{r_n} > 1 - \frac{r_n}{r_m} \]
    if $m<n$, and deduce that $\sum\dfrac{a_n}{r_n}$ diverges.

    \part Prove that
    \[ \frac{a_n}{\sqrt{r_n}} < 2(\sqrt{r_n} - \sqrt{r_{n+1}}) \]
    and deduce that $\sum\dfrac{a_n}{\sqrt{r_n}}$ converges.
  \end{parts}

  \question Prove that the Cauchy product of two absolutely convergent series converges absolutely.

  \question If $\{s_n\}$ is a complex sequence, define its arithmetic means $\sigma_n$ by
  \[ \sigma_n = \frac{s_0 + s_1 + \cdots + s_n}{n+1} \qquad (n=0,1,2,\ldots). \]
  \begin{parts}
    \part If $\lim s_n=s$, prove that $\lim\sigma_n=s$.

    \part Construct a sequence $\{s_n\}$ which does not converge, although $\lim\sigma_n=0$.

    \part Can it happen that $s_n>0$ for all $n$ and that $\limsup s_n=\infty$, although $\lim\sigma_n=0$?

    \part Put $a_n=s_n-s_{n-1}$, for $n\geq1$. Show that
    \[ s_n - \sigma_n = \frac{1}{n+1}\sum_{k=1}^nka_k. \]
    Assume that $\lim (na_n)=0$ and that $\{\sigma_n\}$ converges. Prove that $\{s_n\}$ converges. [This gives a converse of (a), but under the additional assumption that $na_n\to0$.]

    \part Derive the last conclusion from a weaker hypothesis: Assume $M<\infty$, $\abs{na_n}\leq M$ for all $n$, and $\lim\sigma_n=\sigma$. Prove that $\lim s_n=\sigma$, by completing the following outline:

    If $m<n$, then
    \[ s_n - \sigma_n = \frac{m+1}{n-m}(\sigma_n - \sigma_m) + \frac{1}{n-m}\sum_{i=m+1}^n (s_n - s_i). \]
    For these $i$,
    \[ \abs{s_n - s_i} \leq \frac{(n-i)M}{i+1} \leq \frac{(n-m-1)M}{m+2}. \]
    Fix $\varepsilon>0$ and associate with each $n$ the integer $m$ that satisfies
    \[ m \leq \frac{n-\varepsilon}{1+\varepsilon} < m+1. \]
    Then $(m+1)/(n-m)\leq1/\varepsilon$ and $\abs{s_n-s_i}<M\varepsilon$. Hence
    \[ \limsup_{n\to\infty}\abs{s_n-\sigma}\leq M\varepsilon. \]
    Since $\varepsilon$ was arbitrary, $\lim s_n=\sigma$.
  \end{parts}

  \question Definition 3.21 can be extended to the case in which the $a_n$ lie in some fixed $\R^k$. Absolute convergence is defined as convergence of $\sum\abs{\vec{a}_n}$. Show that Theorems 3.22, 3.23, 3.25(a), 3.33, 3.34, 3.42, 3.45, 3.47, and 3.55 are true in this more general setting. (Only slight modifications are required in any of the proofs.)

  \question Fix a positive number $\alpha$. Choose $x_1>\sqrt{\alpha}$, and define $x_2,x_3,x_4,\ldots,$ by the recursion formula
  \[ x_{n+1} = \frac{1}{2}\left( x_n + \frac{\alpha}{x_n} \right). \]
  \begin{parts}
    \part Prove that $\{x_n\}$ decreases monotonically and that $\lim x_n=\sqrt{\alpha}$.

    \part Put $\varepsilon_n=x_n-\sqrt{\alpha}$ and show that
    \[ \varepsilon_{n+1} = \frac{\varepsilon_n^2}{2x_n} < \frac{\varepsilon_n^2}{2\sqrt{\alpha}} \]
    so that, setting $\beta=2\sqrt{\alpha}$,
    \[ \varepsilon_{n+1} < \beta\left( \frac{\varepsilon_1}{\beta} \right)^{2^n} \qquad (n=1,2,3,\ldots). \]

    \part This is a good algorithm for computing square roots, since the recursion formula is simple and the convergence is extremely rapid. For example, if $\alpha=3$ and $x_1=2$, show that $\varepsilon_1/\beta < \frac{1}{10}$ and that therefore
    \[ \varepsilon_5 < 4\cdot10^{-16}, \qquad \varepsilon_6 < 4\cdot10^{-32}. \]
  \end{parts}

  \question Fix $\alpha>1$. Take $x_1>\sqrt{\alpha}$, and define
  \[ x_{n+1} = \frac{\alpha+x_n}{1+x_n} = x_n + \frac{\alpha-x_n^2}{1+x_n}. \]
  \begin{parts}
    \part Prove that $x_1>x_3>x_5>\cdots$.

    \part Prove that $x_2<x_4<x_6<\cdots$.

    \part Prove that $\lim x_n=\sqrt{\alpha}$.

    \part Compare the rapidity of convergence of this process with the one described in Exercise 16.
  \end{parts}

  \question Replace the recursion formula of Exercise 16 by
  \[ x_{n+1} = \frac{p-1}{p}x_n + \frac{\alpha}{p}x_n^{-p+1} \]
  where $p$ is a fixed positive integer, and describe the behaviour of the resulting sequences $\{x_n\}$.

  \question Associate to each sequence $a=\{\alpha_n\}$, in which $\alpha_n$ is 0 or 2, the real number
  \[ x(a) = \sum_{n=1}^\infty \frac{\alpha_n}{3^n}. \]
  Prove that the set of all $x(a)$ is precisely the Cantor set described in Sec. 2.44.

  \question Suppose $\{p_n\}$ is a Cauchy sequence in a metric space $X$, and some subsequence $\{p_{n_i}\}$ converges to a point $p\in X$. Prove that the full sequence $\{p_n\}$ converges to $p$.

  \question Prove the following analogue of Theorem 3.10(b): If $\{E_n\}$ is a sequence of closed nonempty and bounded sets in a \emph{complete} metric space $X$, if $E_n\supset E_{n+1}$, and if
  \[ \lim_{n\to\infty} \diam E_n = 0, \]
  then $\bigcap_1^\infty E_n$ consists of exactly one point.

  \question Suppose $X$ is a nonempty complete metric space, and $\{G_n\}$ is a sequence of dense open subsets of $X$. Prove Baire's theorem, namely, that $\bigcap_1^\infty G_n$ is not empty. (In fact, it is dense in $X$.) \emph{Hint:} Find a shrinking sequence of neighborhoods $E_n$ such that $\cl{E_n}\subset G_n$, and apply Exercise 21.

  \question Suppose $\{p_n\}$ and $\{q_n\}$ are Cauchy sequences in a metric space $X$. Show that the sequence $\{d(p_n,q_n)\}$ converges. \emph{Hint:} For any $m,n$,
  \[d(p_n,q_n) \leq d(p_n,p_m) + d(p_m,q_m) + d(q_m,q_n); \]
  it follows that
  \[ \abs{d(p_n,q_n) - d(p_m,q_m)} \]
  is small if $m$ and $n$ are large.

  \question Let $X$ be a metric space.
  \begin{parts}
    \part Call two Cauchy sequences $\{p_n\}$, $\{q_n\}$ in $X$ \emph{equivalent} if
    \[ \lim_{n\to\infty} d(p_n,q_n) = 0. \]
    Prove that this is an equivalence relation.

    \part Let $X^\ast$ be the set of all equivalence classes so obtained. If $P\in X^\ast$, $Q\in X^\ast$, $\{p_n\}\in P$, $\{q_n\}\in Q$, define
    \[ \Delta(P, Q) = \lim_{n\to\infty} d(p_n,q_n); \]
    by Exercise 23, this limit exists. Show that the number $\Delta(P, Q)$ is unchanged if $\{p_n\}$ and $\{q_n\}$ are replaced by equivalent sequences, and hence that $\Delta$ is a distance function in $X^\ast$.

    \part Prove that the resulting metric space $X^\ast$ is complete.

    \part For each $p\in X$, there is a Cauchy sequence all of whose terms are $p$; let $P_p$ be the element of $X^\ast$ which contains this sequence. Prove that
    \[ \Delta(P_p, P_q) = d(p,q) \]
    for all $p,q\in X$. In other words, the mapping $\varphi$ defined by $\varphi(p)=P_p$ is an isometry (i.e., a distance-preserving mapping) of $X$ into $X^\ast$.

    \part Prove that $\varphi(X)$ is dense in $X^\ast$, and that $\varphi(X)=X^\ast$ if $X$ is complete. By (d), we may identify $X$ and $\varphi(X)$ and thus regard $X$ as embedded in the complete metric space $X^\ast$. We call $X^\ast$ the \emph{completion} of $X$.
  \end{parts}

  \question Let $X$ be the metric space whose points are the rational numbers, with the metric $d(x,y)=\abs{x-y}$. What is the completion of this space? (Compare Exercise 24.)
\end{questions}

%%% Local Variables:
%%% mode: latex
%%% TeX-master: "rudin"
%%% End:
