\section{The Real and Complex Number Systems}

Unless the contrary is explicitly stated, all numbers that are mentioned in these exercises are understood to be real.

\begin{questions}
  \question If $r$ is rational ($r\neq0$) and $x$ is irrational, prove that $r+x$ and $rx$ are irrational.
  \begin{solution}
    If $r+x$ were rational, then so too would be $(-r)+(r+x)=x$. If $rx$ were rational, then since $r\neq0$, $(1/r)\cdot(rx)=x$ would also be rational.

    As $x$ is irrational, it follows that $r+x$ and $rx$ are irrational too.
  \end{solution}

  \question Prove that there is no rational number whose square is 12.
  \begin{solution}
    Suppose there was a rational $r$ satisfying $r^2=12$. Then we may write $r=m/n$ where $m$ and $n$ are integers that are not both multiples of 3. The integers $m,n$ satisfy
    \[ m^2 = 12n^2.. \]
    Hence $m^2$ is a multiple of 3. Since 3 is prime it follows that $m$ is a multiple of 3. Similarly we may deduce that $m$ is a multiple of 2. Hence $m$ is a multiple of 6, so we may write $m=6k$ where $k$ is an integer. This yields
    \[ 36k^2 = 12n^2 \implies 3k^2 = n^2. \]
    Then $n^2$ is a multiple of 3, so that $n$ is a multiple of 3. This contradicts our choice of $m$ and $n$. Hence can be no rational number whose square is 12.
  \end{solution}

  \question Prove Proposition 1.15: The axioms for multiplication imply the following statements.
  \begin{parts}
    \part If $x\neq0$ and $xy=xz$ then $y=z$.

    \part If $x\neq0$ and $xy=x$ then $y=1$.

    \part If $x\neq0$ and $xy=1$ then $y=1/x$.

    \part If $x\neq0$ then $1/(1/x)=x$.
  \end{parts}
  \begin{solution}
    If $x\neq0$ and $xy=xz$, the axioms for multiplication give
    \begin{align*}
      y = 1\cdot y &= \left( \frac{1}{x}\cdot x \right)y = \frac{1}{x}\cdot(xy) \\
                   &= \frac{1}{x}\cdot(xz) = \left(\frac{1}{x}\cdot x\right)z = 1\cdot z = z.
    \end{align*}
    This proves (a). Take $z=1$ in (a) to obtain (b). Take $z=1/x$ in (a) to obtain (c). Since $(1/x)\cdot x=1$, (c) (with $1/x$ in place of $x$, and $x$ in place of $y$) gives (d).
  \end{solution}

  \question Let $E$ be a nonempty subset of an ordered set; suppose $\alpha$ is a lower bound of $E$ and $\beta$ is an upper bound of $E$. Prove that $\alpha\leq\beta$.
  \begin{solution}
    Let $x\in E$, then $\alpha \leq x$, since $\alpha$ is a lower bound of $E$, and $x\leq \beta$, since $\beta$ is an upper bound of $E$. If $\alpha=x$ then from $x\leq\beta$ we see $\alpha\leq\beta$. Similarly if $x=\beta$ then from $\alpha\leq x$ we see $\alpha\leq\beta$.

    If neither $\alpha=x$ nor $x=\beta$ hold, then $\alpha<x$ and $x<\beta$, in which case it follows from the definition of an ordered set that $\alpha<\beta$. Hence in all cases, $\alpha\leq\beta$.
  \end{solution}

  \question Let $A$ be a nonempty set of real numbers which is bounded below. Let $-A$ be the set of all numbers $-x$, where $x\in A$. Prove that
  \[ \inf A = -\sup(-A). \]
  \begin{solution}
    Let $x\in A$, then $-x\in-A$, hence
    \[ -x \leq \sup(-A) \implies -\sup(-A) \leq x. \]
    Therefore $-\sup(-A)$ is a lower bound of $A$. Let $\alpha>-\sup(-A)$, then $-\alpha<\sup(-A)$. Hence $-\alpha$ is not an upper bound of $-A$. All elements of $-A$ have the form $-x$ for some $x\in A$, so there exists some $x\in A$ for which
    \[ -\alpha < -x \implies x < \alpha. \]
    Therefore $\alpha$ is not a lower bound of $A$. It follows that
    \[ \inf A = -\sup(-A). \]
  \end{solution}

  \question Fix $b>1$.
  \begin{parts}
    \part If $m$, $n$, $p$, $q$ are integers, $n>0$, $q>0$, and $r=m/n=p/q$, prove that
    \[ (b^m)^{1/n} = (b^p)^{1/q}. \]
    Hence it makes sense to define $b^r=(b^m)^{1/n}$.
    \begin{solution}
      Let $y=(b^m)^{1/n}$ be the unique positive $n$th root of $b^m$. Then $y^n=b^m$, hence $y^{nq}=b^{mq}$. Since $m/n=p/q$, $mq=np$, therefore $b^{mq}=b^{np}$. It follows that $y^q$ and $b^p$ are both positive $n$th roots of the common number $y^{nq}=b^{np}$. As positive $n$th roots are unique, $y^q=b^p$, and so $y=(b^p)^{1/q}$.
    \end{solution}

    \part Prove that $b^{r+s}=b^rb^s$ is $r$ and $s$ are rational.
    \begin{solution}
      Write $r$ and $s$ with a common positive denominator $d>0$, say $r=m/d$ and $s=n/d$, where $m$, $n$, and $d$ are all integers. Then
      \begin{align*}
        (b^rb^s)^d &= ((b^m)^{1/d}\cdot(b^n)^{1/d})^d \\
                   &= ((b^m)^{1/d})^d\cdot((b^{n})^{1/d})^d = b^mb^n = b^{m+n}.
      \end{align*}
      Therefore $b^rb^s$ is the unique positive $d$th root of $b^{m+n}$, hence
      \[ b^rb^s = (b^{m+n})^{1/d} = b^{(m+n)/d} = b^{r+s}. \]
    \end{solution}

    \part If $x$ is real, define $B(x)$ to be the set of all numbers $b^t$, where $t$ is rational and $t\leq x$. Prove that
    \[ b^r = \sup B(r) \]
    when $r$ is rational. Hence it makes sense to define
    \[ b^x = \sup B(x) \]
    for every real $x$.
    \begin{solution}
      It suffices to show that $b^t\leq b^r$ for every rational $t\leq r$. Then $b^r$ will be an upper bound of $B(r)$, while also being an element of it, so it will follow that it is the least upper bound of $B(r)$.

      Let $r,t$ be rationals with $t\leq r$. Write them with a common positive denominator $d>0$, say $r=m/d$ and $t=n/d$, where $m$, $n$, and $d$ are all integers. Then $m\geq n$, so $m-n\geq0$. An easy induction will verify that $b^{m-n}\geq1$ for any nonnegative integer value of $m-n$. Therefore
      \[ b^m=b^n\cdot b^{m-n}\geq b^n. \]
      Suppose for a contradiction that $b^t>b^r$, i.e. $(b^n)^{1/d}>(b^m)^{1/d}$. It follows by induction that
      \[ ((b^n)^{1/d})^d>((b^m)^{1/d})^d \implies b^n > b^m, \]
      which is a contradiction. Hence $b^t\leq b^r$, and so the result follows.
    \end{solution}

    \part Prove that $b^{x+y}=b^xb^y$ for all real $x$ and $y$.
    \begin{solution}
      Define
      \[ B(x,y)=\{b^rb^s \mid r,s\in\Q, r\leq x, s\leq y\}. \]
      We claim that $b^xb^y=\sup B(x,y)$. Let $r,s\in\Q$ with $r\leq x$ and $s\leq y$, then certainly $0<b^r\leq b^x$ and $0<b^s\leq b^y$. Therefore $b^rb^s\leq b^xb^y$, so $b^xb^y$ is certainly an upper bound of $B(x,y)$. Now notice that for all $r,s\in\Q$ with $r\leq x$ and $s\leq y$, we have
      \[ b^rb^s \leq \sup B(x,y) \implies b^r \leq \frac{1}{b^s} \sup B(x,y). \]
      Hence for every $s\in\Q$ with $s\leq y$, $(1/b^s)\sup B(x,y)$ is an upper bound of $B(x)$. Therefore
      \[ b^x \leq \frac{1}{b^s}\sup B(x,y) \implies b^s \leq \frac{1}{b^x}\sup B(x,y). \]
      Hence $(1/b^x)\sup B(x,y)$ is an upper bound of $B(y)$, so
      \[ b^y \leq \frac{1}{b^x}\sup B(x,y) \implies b^xb^y \leq \sup B(x,y). \]
      Since $b^xb^y$ is an upper bound of $B(x,y)$, we must have $b^xb^y=\sup B(x,y)$.

      Now if $r,s\in\Q$, $r\leq x$, and $s\leq y$, then certainly $r+s\in\Q$ and $r+s\leq x+y$. As $b^rb^s=b^{r+s}$ for all rational $r,s$, it follows that
      \[ B(x,y) \subset B(x+y) \implies b^xb^y \leq b^{x+y}. \]

      Now let us define
      \[ B'(x+y) = \{ b^t \mid t\in\Q, t<x+y\}. \]
      We claim that $b^{x+y}=\sup B'(x+y)$. Clearly $B'(x+y)\subset B(x+y)$, so $b^{x+y}$ is an upper bound of $B'(x+y)$. Let $z<b^{x+y}$. Then $z$ is not an upper bound of $B(x+y)$, so there exists a $t\in\Q$ with $t\leq x+y$ and $z<b^t$. By the archimedian property of $\R$ there is a positive integer $n$ such that
      \[ n(b^t-z) > z(b-1) \implies n > \frac{b-1}{b^tz^{-1} - 1}. \]

      Note that \textbf{7}.(c) can be completed without using \textbf{6}.(d). By \textbf{7}.(c) we have
      \[ b^{1/n} < b^tz^{-1} \implies z < b^tb^{-1/n} = b^{t-(1/n)}. \]
      Now $t-(1/n)<x+y$, hence $b^{t-(1/n)}\in B'(x+y)$. Therefore $z$ is not an upper bound of $B'(x+y)$. It follows that $b^{x+y}$ is the least upper bound of $B'(x+y)$.

      Let $t\in\Q$, $t<x+y$. Then due to the density of $\Q$ in $\R$, there exist $r,s\in\Q$ such that
      \[ x - \frac{1}{2}(x+y-t) < r < x, \qquad y - \frac{1}{2}(x+y-t) < s < y.  \]

      Then $r<x$, $s<y$ and $t<r+s$. Hence $b^t<b^{r+s}=b^rb^s\in B(x,y)$. Hence $b^xb^y$ is an upper bound of $B'(x+y)$, and therefore $b^xb^y\geq b^{x+y}$. It follows that $b^{x+y}=b^xb^y$ for all real $x,y$.
    \end{solution}
  \end{parts}

  \question Fix $b>1$, $y>0$, and prove that there is a unique real $x$ such that $b^x=y$, by completing the following outline. (This is called the logarithm of $y$ to the base $b$.)
  \begin{parts}
    \part For any positive integer $n$, $b^n-1\geq n(b-1)$.
    \begin{solution}
      The result is clear when $n=1$. Suppose the result holds for some positive integer $n$, then since $b^n>1$ we have
      \begin{align*}
        b^{n+1}-1  &= b^n(b-1) + (b^n-1) \\
                   &> (b-1)+n(b-1) \\
                   &= (n+1)(b-1).
      \end{align*}
      By induction it follows that $b^n-1\geq n(b-1)$ for all positive integers $n$.
    \end{solution}

    \part Hence $b-1\geq n(b^{1/n}-1)$.
    \begin{solution}
      Apply part (a) with $b^{1/n}$ in place of $b$.
    \end{solution}

    \part If $t>1$ and $n>(b-1)/(t-1)$, then $b^{1/n}<t$.
    \begin{solution}
      Since $t-1>0$, multiplying both sides of $n>(b-1)/(t-1)$ by $t-1$, and applying (b) yields
      \[ n(t-1) > (b-1) \geq n(b^{1/n}-1) \implies t-1 > b^{1/n} - 1. \]
      Hence $b^{1/n}<t$.
    \end{solution}

    \part If $w$ is such that $b^w<y$, then $b^{w+(1/n)}<y$ for sufficiently large $n$; to see this, apply part (c) with $t=y\cdot b^{-w}$.
    \begin{solution}
      By the archimedian property of $\R$, there exists a positive integer $N$ such that
      \[ N(y-b^w) > b^w(b-1) \implies N > \frac{b-1}{y\cdot b^{-w} - 1}. \]
      Then for all $n\geq N$, if we let $t=y\cdot b^{-w}$ then $n>(b-1)/(t-1)$, so by (c)
      \[ b^{1/n} < y\cdot b^{-w} \implies b^{w+(1/n)} = b^{1/n}b^w < y. \]
    \end{solution}

    \part If $b^w>y$, then $b^{w-(1/n)}>y$ for sufficiently large $n$.
    \begin{solution}
      We have $b^{-w}<y^{-1}$, so  from the previous part, $b^{-w+(1/n)}<y^{-1}$ for sufficiently large $n$. Hence $b^{w-(1/n)}>y$ for sufficiently large $n$.
    \end{solution}

    \part Let $A$ be the set of all $w$ such that $b^w<y$, and show that $x=\sup A$ satisfies $b^x=y$.
    \begin{solution}
      Suppose $b^x<y$. By (d) there is a positive $n$ such that $b^{x+(1/n)}<y$, so that $x+(1/n)\in A$, which contradicts $x$ being an upper bound of $A$.

      Suppose $b^x>y$. By (e) there is a positive $n$ with $b^{x-(1/n)}>y$, so that $b^{x-(1/n)}>b^w$ for all $w\in A$. Therefore $b^w$ is not an upper bound of $B(x-(1/n))$, for all $w\in A$. So for all $w\in A$ there is a $t\in \Q$ with $t\leq x-(1/n)$ such that
      \[ b^t > b^w \implies t > w. \]

      By the density of $\Q$ in $\R$ it follows that
      \[ x-(1/n)=\sup\{t\in\Q \mid t\leq x-(1/n)\}. \]
      Therefore $x-(1/n)\geq w$ for all $w\in A$, hence $x-(1/n)$ is an upper bound of $A$. This contradicts $x$ being the least upper bound of $A$. The only possibility which remains is $b^x=y$.
    \end{solution}

    \part Prove that this $x$ is unique.
    \begin{solution}
      If $x>z$, then there exists a $t\in\Q$ with $0<t<x-z$, so that $b^t<b^{x-z}$. Since $t>0$ then $b^t>b^0=1$, hence
      \[ b^x = b^{x-z}b^z > b^tb^z > b^z. \]
      Similarly if $x<z$ then $b^x<b^z$. So if $b^x=y=b^z$ we must have $x=z$.
    \end{solution}
  \end{parts}

  \question Prove that no order can be defined in the complex field that turns it into an ordered field. \emph{Hint:} $-1$ is a square.
  \begin{solution}
    In an ordered field, if $x\neq0$, then $x^2>0$. For, if $x>0$ then $x^2>0$, and if $x<0$ then $-x>0$, meaning $x^2=(-x)^2>0$. Therefore $1=1^2>0$, and hence $-1<0$. However if we could define an order on the complex field which turns it into an ordered field, then
    \[ -1 = i^2 > 0, \]
    which contradicts $-1<0$. Hence there is no way of defining such an order.
  \end{solution}

  \question Suppose $z=a+bi$, $w=c+di$. Define $z<w$ if $a<c$, and also if $a=c$ but $b<d$. Prove that this turns the set of all complex numbers into an ordered set. (This type of order relation is called a \emph{dictionary order}, or \emph{lexicographic order}, for obvious reasons.) Does this ordered set have the least-upper-bound property?
  \begin{solution}
    Let $z=a+bi$, $w=c+di$. Precisely one of $a<c$, $a=c$ and $c<a$ holds. In the first case, $z<w$, $z\neq w$ and $w\not<z$. In the last case, $w<z$, $z\neq w$ and $z\not>w$.

    If $a=c$, then precisely one of $b<d$, $b=d$ and $d<b$ holds. In the first case $z<w$, $z<w$ and $w\not<z$. In the second case $z=w$, $z\not<w$ and $w\not<z$. In the last case $w<z$, $z\neq w$ and $z\not<w$. In all cases, precisely one of $z<w$, $z=w$, $w<z$ holds.

    Now suppose $y=u+iv$ and that $z<w$ and $w<y$. If $a<c$, then since $w<y$ we have $c\leq u$. Hence $a<u$ and so $z<y$. The only other possibility is when $a=c$. If $c<u$ then $a<u$ and so $z<y$.

    Otherwise $c=u$, and so $a=u$. Since $z<w$ and $w<y$ we must have $b<d$ and $d<v$, hence $b<v$. Therefore $z<y$ in this case also. Therefore if $z<w$ and $w<y$, then $z<y$. Hence this order turns the set of all complex numbers into an ordered set.

    This ordered set does not have the least-upper-bound property. For example, define $E=\{0+vi \mid v\in\R\}$. Then $E$ is nonempty and bounded above by $1$ (as $0+vi<1+0i$ for every $v\in\R$).

    Suppose that $a+bi$ were a least-upper-bound of $E$, then certainly $0\leq a$. If $0<a$, then $0+vi<a/2+0i$ for all $v\in\R$, and $a/2+0i<a+bi$, which contradicts $a+bi$ being the least upper bound. Hence $E$ has no least-upper-bound.
  \end{solution}

  \question Suppose $z=a+bi$, $w=u+iv$, and
  \[ a = \left( \frac{\abs{w} + u}{2} \right)^{1/2}, \qquad b = \left( \frac{\abs{w} - u}{2} \right)^{1/2}. \]
  Prove that $z^2=w$ if $v\geq0$ and that $(\conj{z})^2$ if $v\leq0$. Conclude that every complex number (with one exception!) has two complex square roots.
  \begin{solution}
    Note that $x^{1/2}y^{1/2}$ is the unique positive square root of $xy$, hence $x^{1/2}y^{1/2}=(xy)^{1/2}$. We then have the following calculation
    \begin{align*}
      z^2 &= a^2-b^2 + 2abi \\
          &= \frac{\abs{w}+u}{2} - \frac{\abs{w}-u}{2} + 2i\left( \frac{\abs{w}+u}{2} \right)^{1/2}\left( \frac{\abs{w}-u}{2} \right)^{1/2} \\
          &= u + i(\abs{w}^2-u^2)^{1/2} \\
          &= u + i(u^2+v^2-u^2)^{1/2} \\
          &= u + i(v^2)^{1/2}.
    \end{align*}
    If $v\geq0$, then $v$ is the unique positive square root of $v^2$, hence $(v^2)^{1/2}=v$, and so in this case $z^2=w$. A similar calculation gives $(\conj{z})^2=u-i(v^2)^{1/2}$, and if $v\leq0$ then $-v$ is the unique positive square root of $v^2$, hence $(v^2)^{1/2}=-v$, and so in this case $(\conj{z})^2=w$.

    So we have demonstrated that for every complex number $w$, there exists a complex number $z$ for which $z^2=w$. It is clear that $z=0$ iff $w=0$, so provided $w\neq0$ we actually have two complex square roots, namely $z$ and $-z$. Moreover, let $x$ be a complex number with $x^2=w$. Then we have
    \[ x^2 - z^2 = 0 \implies (x-z)(x+z) = 0. \]
    So it follows that either $x=z$ or $x=-z$. Hence every complex number has exactly two complex square roots, with the exception of 0 which has exactly one.
  \end{solution}

  \question If $z$ is a complex number, prove that there exists an $r\geq0$ and a complex number $w$ with $\abs{w}=1$ such that $z=rw$. Are $w$ and $r$ always uniquely determined by $z$?
  \begin{solution}
    If $z\neq0$, then $\abs{z}\neq0$. In this case, define $r=\abs{z}$ and $w=z/\abs{z}$. Then it is clear that $z=rw$, and $\abs{w}=\abs{z}/\abs{z}=1$. If $z=0$ then define $r=0$ and $w=1$, then $\abs{w}=1$ and $z=0\cdot1=rw$.

    The number $r$ is uniquely determined by $z$, in fact we must have $r=\abs{z}$. For, $\abs{z}=\abs{rw}=\abs{r}$, and since $r\geq0$ it follows that $r=\abs{z}$. However $w$ is not uniquely determined by $z$. For example $0=0\cdot1=0\cdot(-1)$, and $\abs{1}=\abs{-1}=1$.
  \end{solution}

  \question If $z_1,\ldots,z_n$ are complex, prove that
  \[ \abs{z_1+z_2+\cdots+z_n} \leq \abs{z_1}+\abs{z_2}+\cdots+\abs{z_n}. \]
  \begin{solution}
    We prove this by induction on the number of variables. Note that we already have the triangle inequality for triangle numbers. The inequality is clear when $n=1$. Suppose that it holds for some $n\geq1$, then
    \begin{align*}
      \abs{z_1+z_2\cdots+z_n+z_{n+1}} &\leq \abs{z_1+z_2+\cdots+z_n} + \abs{z_{n+1}} \tag{Triangle inequality} \\
                                      &\leq \abs{z_1} + \abs{z_2} + \cdots + \abs{z_n} + \abs{z_{n+1}} \tag{Inductive hypothesis}
    \end{align*}
    The result follows by induction.
  \end{solution}

  \question If $x,y$ are complex, prove that
  \[ \abs{\abs{x} - \abs{y}} \leq \abs{x-y}. \]
  \begin{solution}
    We have
    \[ \abs{x} \leq \abs{x-y} + \abs{y} \implies \abs{x} - \abs{y} \leq \abs{x-y}. \]
    Since $\abs{x-y}=\abs{y-x}$, by symmetry $\abs{y}-\abs{x}\leq\abs{x-y}$ also. Because $\abs{\abs{x}-\abs{y}}$ is one of $\abs{x}-\abs{y}$, $\abs{y}-\abs{x}$, it follows that
    \[ \abs{\abs{x}-\abs{y}} \leq \abs{x-y}. \]
  \end{solution}

  \question If $z$ is a complex number such that $\abs{z}=1$, that is, such that $z\conj{z}=1$, compute
  \[ \abs{1+z}^2 + \abs{1-z}^2. \]
  \begin{solution}
    Let $z=a+bi$, then $a^2+b^2=1$, so
    \begin{align*}
      \abs{1+z}^2 + \abs{1-z}^2 &= (1+a)^2+b^2 + (1-a)^2+b^2 \\
                                &= 2 + 2(a^2+b^2) \\
                                &= 4.
    \end{align*}
  \end{solution}

  \question Under what conditions does equality hold in the Schwarz inequality?
  \begin{solution}
    The Schwarz inequality states that if $a_1,a_2\ldots,a_n$ and $b_1,b_2\ldots,b_n$ are complex numbers, then
    \[ \abs*{\sum_{j=1}^n a_j\conj{b_j}}^2 \leq \sum_{j=1}^n \abs{a_j}^2 \sum_{j=1}^n \abs{b_j}^2. \]

    We claim that equality holds precisely when either $b_1=b_2=\cdots=b_n=0$, or there exists a $\lambda\in\C$ such that $a_1=\lambda b_1$, $a_2=\lambda b_2$ \ldots, $a_n=\lambda b_n$. Clearly equality holds in the first case, so suppose not all of $b_1,b_2,\ldots,b_n$ are zero. The given proof of the inequality defines
    \[ A = \sum_{j=1}^n \abs{a_j}^2, \ B = \sum_{j=1}^n \abs{b_j}^2, \ C = \sum_{j=1}^n a_j\conj{b_j}, \]
    and proceeds to show that
    \[ \sum_{i=1}^n \abs{Ba_i-Cb_i}^2 = B(AB-\abs{C}^2). \]
    Since $B>0$ it follows that $\abs{C}^2\leq AB$, which is precisely the inequality. Now equality holds if and only if $Ba_i=Cb_i$ for each $i=1,\ldots,n$. Since $B>0$ we can define $\lambda = C/B$, and then $a_i=\lambda b_i$ for each $i=1,\ldots,n$. Conversely, suppose there is a $\lambda\in\C$ such that $a_i=\lambda b_i$ for each $i=1,\ldots,n$. Then
    \[ Ba_i = a_i\sum_{j=1}^n \abs{\lambda a_j}^2 = \conj{\lambda}\lambda a_i \sum_{j=1}^n a_j\conj{a_j} = b_i\sum_{j=1}^n a_j(\conj{\lambda a_j}) = Cb_i, \]
    and so equality holds. This proves the claim.
  \end{solution}

  \question Suppose $k\geq3$, $\vec{x}, \vec{y}\in\R^k$, $\abs{\vec{x}-\vec{y}}=d>0$, and $r>0$. Prove:
  \begin{parts}
    \part If $2r>d$, there are infinitely many $\vec{z}\in\R^k$ such that
    \[ \abs{\vec{z} - \vec{x}} = \abs{\vec{z} - \vec{y}} = r. \]

    \part If $2r=d$, there is exactly one such $\vec{z}$.

    \part If $2r<d$, there is no such $\vec{z}$.
  \end{parts}
  How must these statements be modified if $k$ is 2 or 1?
  \begin{solution}
    Suppose that $\vec{z}$ satisfies $\abs{\vec{z}-\vec{x}}=\abs{\vec{z}-\vec{y}}=r$. Then the Schwarz inequality gives
    \begin{align*}
      d^2 = \abs{\vec{x}-\vec{y}}^2 &= \abs{(\vec{x}-\vec{z}) + (\vec{z}-\vec{y})}^2 \\
                                      &= \abs{\vec{x}-\vec{z}}^2 + 2(\vec{x}-\vec{z})\cdot(\vec{z}-\vec{y}) + \abs{\vec{z}-\vec{y}}^2 \\
                                      &\leq \abs{\vec{x}-\vec{z}}^2 + 2\abs{\vec{x}-\vec{z}}\abs{\vec{z}-\vec{y}} + \abs{\vec{z}-\vec{y}}^2 \\
                                      &= (2r)^2.
    \end{align*}
    Hence if there are any $\vec{z}$ satisfying the equation, then $d\leq2r$, which answers (c).

    If $2r=d$ then equality must hold in the above application of the Schwarz inequality. By \textbf{15}., since $r>0$ we conclude there is a $\lambda\in\R$ with $\vec{x}-\vec{z}=\lambda(\vec{z}-\vec{y})$. Then
    \[ r = \abs{\lambda(\vec{z}-\vec{y})} = \abs{\vec{z}-\vec{y}} \implies \abs{\lambda}=1. \]
    If $\lambda=-1$, then $\vec{x}=\vec{y}$, contradicting $d>0$. Hence $\lambda=1$, and so $\vec{z}=(\vec{x}+\vec{y})/2$. So if $2r=d$, there is precisely one such $\vec{z}$, which answers (b).

    The statements in (b) and (c) need not be modified if $k$ is 2 or 1.
  \end{solution}

  \question Prove that
  \[ \abs{\vec{x} + \vec{y}}^2 + \abs{\vec{x} - \vec{y}}^2 = 2\abs{\vec{x}}^2 + 2\abs{\vec{y}}^2 \]
  if $\vec{x}\in\R^k$ and $\vec{y}\in\R^k$. Interpret this geometrically, as a statement about parallelograms.
  \begin{solution}
    \begin{align*}
      \abs{\vec{x}+\vec{y}}^2 + \abs{\vec{x}-\vec{y}}^2 &= (\vec{x}+\vec{y})\cdot(\vec{x}+\vec{y}) + (\vec{x}-\vec{y})\cdot(\vec{x}-\vec{y}) \\
                                                            &= (\vec{x}\cdot\vec{x} + 2\vec{x}\cdot\vec{y} + \vec{y}\cdot\vec{y}) + (\vec{x}\cdot\vec{x} - 2\vec{x}\cdot\vec{y} + \vec{y}\cdot\vec{y}) \\
                                                            &= 2\vec{x}\cdot\vec{x} + 2\vec{y}\cdot\vec{y} \\
                                                            &= 2\abs{\vec{x}}^2 + 2\abs{\vec{y}}^2.
    \end{align*}
  \end{solution}

  \question If $k\geq2$ and $\vec{x}\in\R^k$, prove that there exists $\vec{y}\in\R^k$ such that $\vec{y}\neq\vec{0}$ but $\vec{x}\cdot\vec{y}=0$. Is this also true if $k=1$?
  \begin{solution}
    Suppose that $\vec{x}=(x_1,x_2,\ldots,x_k)$. If any one of the $x_j$ is equal to 0, then we may define $\vec{y}=(y_1,y_2,\ldots,y_k)$, where $y_j=1$ and $y_i=0$ whenever $i\neq j$. As $k\geq2$, $\vec{y}\neq\vec{0}$, and
    \[ \vec{x}\cdot\vec{y} = \sum_{i=1}^k x_iy_i = x_jy_j + \sum_{i\neq j} x_iy_i = 0\cdot1 + \sum_{i\neq j} x_i\cdot0 = 0. \]
    Otherwise all of $x_1,x_2,\ldots,x_k$ are nonzero. Since $k\geq 2$, in particular $x_1$ and $x_2$ are nonzero. Hence if we define $\vec{y}=(y_1,y_2,\ldots,y_k)$ by $y_1=x_2$, $y_2=-x_1$ and $y_i=0$ if $i>2$, then $\vec{y}\neq\vec{0}$. Moreover
    \[ \vec{x}\cdot\vec{y} = \sum_{i=1}^k x_iy_i = x_1x_2-x_2x_1 + \sum_{i=3}^k x_i\cdot0 = 0. \]
    So if $k\geq2$, for every $\vec{x}\in\R^k$ there exists $\vec{y}\in\R^k$ such that $\vec{y}\neq\vec{0}$ but $\vec{x}\cdot\vec{y}=0$. However if $k=1$ then we can take $x=1$, then whenever $x\cdot y=0$ we must have $y=0$, therefore the same result is not also true in this case.
  \end{solution}

  \question Suppose $\vec{a}\in\R^k$, $\vec{b}\in\R^k$. Find $\vec{c}\in\R^k$ and $r>0$ such that
  \[ \abs{\vec{x}-\vec{a}} = 2\abs{\vec{x}-\vec{b}} \]
  if and only if $\abs{\vec{x}-\vec{c}}=r$.
  (\emph{Solution:} $3\vec{c}=4\vec{b}-\vec{a}$, $3r=2\abs{\vec{b}-\vec{a}}$.)
  \begin{solution}
    We have the following equality
    \begin{align*}
      4\abs{\vec{x}-\vec{b}}^2 - \abs{\vec{x}-\vec{a}}^2 &= 4\abs{\vec{x}}^2 - 8\vec{x}\cdot\vec{b} + 4\abs{\vec{b}}^2 - \abs{\vec{x}}^2 + 2\vec{x}\cdot\vec{a} - \abs{\vec{a}}^2 \\
                                                             &= 3\abs{\vec{x}}^2 - 2\vec{x}\cdot(4\vec{b} - \vec{a}) + 4\abs{\vec{b}}^2 - \abs{\vec{a}}^2 \\
                                                             &= 3\abs*{\vec{x}-\frac{1}{3}(4\vec{b}-\vec{a})}^2 - \frac{1}{3}\abs{4\vec{b}-\vec{a}}^2 + 4\abs{\vec{b}}^2 - \abs{\vec{a}}^2 \\
                                                             &= 3\abs*{\vec{x}-\frac{1}{3}(4\vec{b}-\vec{a})}^2 - \frac{16}{3}\abs{\vec{b}}^2 + \frac{8}{3}\vec{b}\cdot\vec{a} - \frac{1}{3}\abs{\vec{a}}^2 + 4\abs{\vec{b}}^2 - \abs{\vec{a}}^2 \\
                                                             &= 3\abs*{\vec{x}-\frac{1}{3}(4\vec{b}-\vec{a})}^2 - \frac{4}{3}\abs{\vec{b}}^2 + \frac{8}{3}\vec{b}\cdot\vec{a} - \frac{4}{3}\abs{\vec{a}}^2 \\
                                                             &= 3\abs*{\vec{x}-\frac{1}{3}(4\vec{b}-\vec{a})}^2 - \frac{4}{3}\abs{\vec{b}-\vec{a}}^2.
    \end{align*}
    Hence $\abs{\vec{x}-\vec{a}}=2\abs{\vec{x}-\vec{b}}$ if and only if $\abs{\vec{x}-\frac{1}{3}(4\vec{b}-\vec{a})}=\frac{2}{3}\abs{\vec{b}-\vec{a}}$. So if we set
    \[ \vec{c}=\frac{1}{3}(4\vec{b}-\vec{a}),\quad r=\frac{2}{3}\abs{\vec{b}-\vec{a}}, \]
    then $\abs{\vec{x}-\vec{a}}=2\abs{\vec{x}-\vec{b}}$ if and only if $\abs{\vec{x}-\vec{c}}=r$.
  \end{solution}

  \question With reference to the Appendix, suppose that property (III) were omitted from the definition of a cut. Keep the same definitions of order and addition. Show that the resulting ordered set has the least-upper-bound property, that addition satisfies the axioms (A1) to (A4) (with a slightly different zero-element!) but that (A5) fails.
\end{questions}

%%% Local Variables:
%%% mode: latex
%%% TeX-master: "rudin"
%%% End:
